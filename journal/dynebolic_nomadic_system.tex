\usemodule[pre-01]
\usemodule[tikz]
\usemodule[newmat]
\setupinteraction [state=start]
\starttext
\TitlePage { dyne:bolic nomadic operating system
\blank[3*medium]
\tfa Jaromil - media artist and developer
 \blank[2*medium]
  \tfa July 2008}\Topic {Free Software Movement}

\placefigure[][]{}{\externalfigure[images/floating_gnu.jpg]}


Started  in  1984 by  Richard  Stallman,  with  help by  Eben  Moglen,
drafting the GNU General Public License, granting users the rights to:

\startitemize
\item Run for any purpose
\item Study and adapt
\item Redistribute
\item Distribute modifications
\stopitemize


\Topic {Dyne.org network}

\placefigure[][]{}{\externalfigure[images/dyne.png]}

Started  in  2000 publishing  {\em low-consumption}  software creations  for
broadcasting and freedom of speech, granting users with the rights to:

\startitemize
\item Promote the idea and practice of open source knowledge sharing
\item Open the participation to on-line and on-site communities
\item Foster employment of FOSS in artistic creation
\item Support FOSS development, also when non-profitable
\stopitemize


\Topic {Theoretical background}

\placefigure[][]{}{\externalfigure[images/logolibertelight.jpg]}

\startitemize
\item Collaboration instead of competition
\item No strings attached to marketed products
\item Ownership of production means
\item Global knowledge for local economies

The best  result will  come from everybody  in the group  doing what's
best for himself, and the group. ({\em John Nash, Nobel in 1994})
\stopitemize


\Topic {Dyne:OS implementation}

\placefigure[][]{}{\externalfigure[images/dynebolic.png]}

The {\bf dyne:bolic} GNU/Linux liveCD multimedia operating system, developed
from scratch since 2001 with the following focus:

\startitemize
\item Ease of use, non invasive installation that co-exists with other systems
\item Recycling of existing infrastructure, support for game consoles
\item Oriented to production and not only fruition of media
\item Enforcing privacy of users and independent distribution of information
\item Self contained development toolkit
\stopitemize


\Topic {Dyne:OS in the Makrolab}

\placefigure[][]{}{\externalfigure[images/jaromil_smilzo_newmark_in_makrolab.jpg]}

No production  or merchandising was ever planned  for the distribution
of this system, still as of  today we count more than {\bf 1 million copies}
on printed CDs all around the world.

\startitemize
\item Community sharing
\item Local business development
\item Printed magazines, also those focusing on security and gaming
\stopitemize


\Topic {Dyne:OS cultural impact}

One  of  the  few  {\bf 100\%  free}  (libre,  or  in  other  words  entirely
freedom-respecting) GNU/Linux  distributions that are  recommended and
mirrored by  the Free Software Foundation worldwide,  along with BLAG,
gNewSense, Ututo, Musix and GNUstep.

\blank[medium]\hrule\blank[medium]

The Independent  UK lists us  as {\bf Top 10  open source project  in 2005},
besides other projects as  Wikipedia, Sourceforge, XVid, Gimp, Apache,
Mediawiki, Firefox, OpenOffice:

\ldots 

\startquotation
Most of the world's computers  use Microsoft Windows as the operating
system. Despite  its popularity, Windows  does have drawbacks  - cost
and security holes,  for starters.  There is a  free alternative: the
geeky Linux/GNU.   In the past it  was known as reliable  but hard to
use. {\bf Dyne:bolic is a multimedia studio  on a CD that you simply pop
into  any computer and  start it  up}, instantly  turning it  into a
Linux/GNU system without affecting existing things on your computer.
\stopquotation



\Topic {Dyne:OS cultural impact}

\startitemize
\item Date: Sun, 30 Nov 2003 20:10:37 -0500
\item From: {\bf Jon maddog Hall} \switchtobodyfont[small]maddog at li.org\switchtobodyfont[big]
\item To: discuss at gnhlug.org
\item Subject: If you have not seen Dynebolic, you have not seen\ldots
\stopitemize 

\ldots 

\startquotation
I looked at this distribution six months ago, but as impressive as it
was then, it has gotten even better.
\stopquotation

\startquotation
Take a look.
\stopquotation

\startquotation
md
\stopquotation

\blank[medium]\hrule\blank[medium]

\startquotation
``Jaromil's contribution to the Free Software community is immense. We
appreciate  his   work  from  India.''   -   {\bf Frederick  FN  Noronha,
BytesForAll}. June 24, 2008*
\stopquotation



\Topic {Thanks!}

We  hope to  survive  and do  more, we  have  no funding  but lots  of
courage, contact us if you think you can help.

\placefigure[][]{}{\externalfigure[images/dyne-big.png]}

Jaromil's musings on \useURL[aa][http://jaromil.dyne.org][][http://jaromil.dyne.org] \from[aa]

A thousand flowers will blossom!



\stoptext
