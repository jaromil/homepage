\usemodule[pre-01]
\usemodule[tikz]
\usemodule[newmat]
\setupinteraction [state=start]
\starttext
\TitlePage { Creazione dal basso e P2P
\blank[3*medium]
\tfa Jaromil
 \blank[2*medium]
  \tfa May 2009, Roma}\Topic {The Lies of ``Free'' Market}

The {\bf Laws  of Free-Trade} dictate that  when you exchange  money for the
purchase  of  any item,  that  item  belongs  to you  without  strings
attached.

\placefigure[][]{}{\externalfigure[images/iphone.jpg]}

{\bf Mobile  communication  objects sold  worldwide  are restricted}.   Such
mobile  communication devices constitute  nowadays the  widest network
around the globe, mostly used by citizens for private communications.



\Topic {The Industrial Involution of Art}

\startitemize
\item {\bf Who  owns  what}?   In  digitally  produced  contemporary  art  the
tool-makers (and not the artists!)  own the rights to reproduce the
artworks.

\item {\bf Uniformed design}.   In a  globally connected  world  uniformity is
marketed  better than  variety  (simplifying cognitive  processes):
uncommon ground is seen as  a threat, massive outreach is forced by
flattening perception.

\item {\bf Mis-creative  industries}.  Improvised  social  forms built  around
close-knit  networks and  non-conventional human  relationships are
substituted by  the absorption  of human emotion  into bureaucratic
channels:  the emergence  of a  social coolness,  an  efficiency of
feeling. \footnote{For an extensive  analysis on the topic see "The  Next Idea of the
Artist" essay  by Rana Dasgupta (2008) published  on the catalogue
of the Liverpool Biennial}

\item {\bf Futurist  corpses}.   Ideology  and advertising  have  exalted  the
permanent mobilisation  of the  productive and nervous  energies of
humankind towards profit and war. \footnote{The Post-Futurist Manifesto. Franco Berardi, February 2009}
\stopitemize

\Topic {At the End of Capitalism}

\startitemize
\item Is it DOOM for content  creators?

\item How is  unrestricted copying anything but a  bane for people trying
to make a living while making media?

\item Is there something as a {\bf Peer to Peer Economy}?

\item Think about it.

\item It is {\bf not} barter. Micro-payments maybe help.

\item Think TWICE!

\item Imagine {\bf local economies}. Connect them.

\item Let the people join. Give broad access to production.

\item Redistribute the activity.  Let neighbours take care of each other.

\item Dis-corporate production. {\bf REP RAP}.
\stopitemize


\Topic {The Rise of new Values}

\startitemize
\item {\bf Refuse Scarcity}. Abundance is not our enemy, it is our friend.

\item {\bf Refuse   the  Big  Success},   the  Big  Audience,  the   Big  Mass
Media. Connect  specific communities with  low-cost and low-latency
distribution

\item {\bf Refuse the Broadcast  Quality}. Embrace modesty, publish unfinished,
involve  people,  work   enthusiastically  with  what's  available.
Follow the vector of quality, don't pursue its mirage.

\item {\bf Recycle!} modular architectures, generic infrastuctures.

\item RE-USE! {\bf BRICO BRICO}
\stopitemize

\placefigure[][]{}{\externalfigure[images/bricolabs-sm.jpg]}


\Topic {Successful GNU stories}

\placefigure[][]{}{\externalfigure[images/floating_gnu.jpg]}

Back in  1984 Richard Stallman, with  help by Eben  Moglen and others,
drafted the GNU General Public License, granting users the rights to:

\startitemize
\item Run for any purpose
\item Study and adapt
\item Redistribute
\item Distribute modifications
\stopitemize


\Topic {Tagtool}

\placefigure[][]{}{\externalfigure[images/tagtool.jpg]}

The Tagtool is an open source instrument for drawing and animation in a live performance situation.

The  project  aims   to  explore  digital  drawing  as   a  means  for
communication - on stage, on the street and over the Internet.

The  Tagtool is  in  use as  a  visual instrument  in theatres,  youth
centers, at jam sessions with musicians and for performances in public
spaces. It  serves as a VJ  tool, a creative video  game for children,
and an intuitive tool for creating animation.


\Topic {Reactable}

\placefigure[][]{}{\externalfigure[images/reactable.jpg]}

The Reactable uses a so  called tangible interface, where the musician
controls the  system by manipulating tangible  objects.

The instrument is based on a translucent and luminous round table, and
by putting these  pucks on the Reactable surface,  by turning them and
connecting  them  to  each  other, performers  can  combine  different
elements like synthesizers, effects,  sample loops or control elements
in order to create a unique and flexible composition.


\Topic {Arduino}

\placefigure[][]{}{\externalfigure[images/arduino.jpg]}

Arduino is  an open-source  electronics prototyping platform  based on
flexible,  easy-to-use  hardware   and  software.

It's intended for artists, designers, hobbyists, and anyone interested
in creating interactive objects or environments.

Arduino can sense the environment by receiving input from a variety of
sensors and can affect its surroundings by controlling lights, motors,
and other actuators.



\Topic {dyne:bolic live CD}

\placefigure[][]{}{\externalfigure[images/dynebolic.png]}

The {\bf dyne:bolic} GNU/Linux liveCD multimedia operating system, developed
from scratch since 2001 with the following focus:

\startitemize
\item Ease of use, non invasive installation that co-exists with other systems
\item Recycling of existing infrastructure, support for game consoles
\item Oriented to production and not only fruition of media
\item Enforcing privacy of users and independent distribution of information
\item Self contained development toolkit
\stopitemize


\Topic {dyne:bolic in the Makrolab}

\placefigure[][]{}{\externalfigure[images/jaromil_smilzo_newmark_in_makrolab.jpg]}

No production  or merchandising was ever planned  for the distribution
of this system, still as of  today we count more than {\bf 1 million copies}
on printed CDs all around the world.

\startitemize
\item Community sharing
\item Local business development
\item Printed magazines, also those focusing on security and gaming
\stopitemize


\Topic {dyne:bolic cultural impact}

One  of  the  few  {\bf 100\%  free}  (libre,  or  in  other  words  entirely
freedom-respecting) GNU/Linux  distributions that are  recommended and
mirrored by  the Free Software Foundation worldwide,  along with BLAG,
gNewSense, Ututo, Musix and GNUstep.

\blank[medium]\hrule\blank[medium]

The Independent  UK lists us  as {\bf Top 10  open source project  in 2005},
besides other projects as  Wikipedia, Sourceforge, XVid, Gimp, Apache,
Mediawiki, Firefox, OpenOffice:

\ldots 

\startquotation
Most of the world's computers  use Microsoft Windows as the operating
system. Despite  its popularity, Windows  does have drawbacks  - cost
and security holes,  for starters.  There is a  free alternative: the
geeky Linux/GNU.   In the past it  was known as reliable  but hard to
use. {\bf Dyne:bolic is a multimedia studio  on a CD that you simply pop
into  any computer and  start it  up}, instantly  turning it  into a
Linux/GNU system without affecting existing things on your computer.
\stopquotation



\Topic {The Free Loop of Creation}

An ideal {\em Semantic Square} (Greimas' Rectangle) to play with:

\placefigure[][]{}{\externalfigure[images/loop_of_creation.png]}




\Topic {Salaam/Shalom/Shanthi/Dorood/Peace/Pace}

\placefigure[][]{}{\externalfigure[images/dyne-big.png]}

Netherlands Media Art Institute (NIMK) - R\&D

Jaromil's musings on \useURL[aa][http://jaromil.dyne.org/journal][][http://jaromil.dyne.org/journal] \from[aa]

HINEZUMI / Freaknet Medialab / Poetry Hacklab

Thanks, a thousand flowers will blossom!



\stoptext
